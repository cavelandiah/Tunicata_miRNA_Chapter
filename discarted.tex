\subsection{miRNA identification and validation}
\subsubsection{Pasquinatelli2000.pdf}
\cite{Pasquinelli2000}
\textit{Two small RNAs regulate the timing of Caenorhabditis elegans development. Transition from the first to the second larval stage fates requires the 22-nucleotide lin-4 RNA, and transition from late larval to adult cell fates requires the 21-nucleotide let-7 RNA. The lin-4 and let-7 RNA genes are not homologous to each other, but are each complementary to sequences in the 3' untranslated regions of a set of protein-coding target genes that are normally negatively regulated by the RNAs. Here we have detected let-7 RNAs of approximately 21 nucleotides in samples from a wide range of animal species, including vertebrate, ascidian, hemichordate, mollusc, annelid and arthropod, but not in RNAs from several cnidarian and poriferan species, Saccharomyces cerevisiae, Escherichia coli or Arabidopsis. We did not detect lin-4 RNA in these species. We found that let-7 temporal regulation is also conserved: let-7 RNA expression is first detected at late larval stages in C. elegans and Drosophila, at 48 hours after fertilization in zebrafish, and in adult stages of annelids and molluscs. The let-7 regulatory RNA may control late temporal transitions during development across animal phylogeny.}

\subsubsection{Legendere2004.pdf}
\cite{Legendre2005}
\textit{MOTIVATION: MicroRNAs (miRNA) are essential 21-22 nt regulatory RNAs produced from larger hairpin-like precursors. Local sequence alignment tools such as BLAST are able to identify new members of known miRNA families, but not all of them. We set out to estimate how many new miRNAs could be recovered using a profile-based strategy such as that implemented in the ERPIN program. RESULTS: We constructed alignments for 18 miRNA families and performed ERPIN searches on animal genomes. Results were compared to those of a WU-BLAST search at the same E-value cutoff. The two combined approaches produced 265 new miRNA candidates that were not found in miRNA databases. About 17{\%} of hits were ERPIN specific. They showed better structural characteristics than BLAST-specific hits and included interesting candidates such as members of the miR-17 cluster in Tetraodon. Profile-based RNA detection will be an important complement of similarity search programs in the completion of miRNA collections.}

\subsubsection{Satdler2005.pdf}
\cite{Missal2005} 
\textit{The analysis of animal genomes showed that only a minute part of their DNA codes for proteins. Recent experimental results agree, however, that a large fraction of these genomes are transcribed and hence are probably functional at the RNA level. A computational survey of vertebrate genomes has predicted thousands of previously unknown ncRNAs with evolutionarily conserved secondary structures. Extending these comparative studies beyond vertebrates is difficult, however, since most ncRNAs evolve quickly at the sequence level while conserving their characteristic secondarystructures. RESULTS: We report on a computational screen of structured ncRNAs in the urochordate lineage based on a comparison of the genomic data from Ciona intestinalis, Ciona savignyi and Oikopleura dioica. We predict {\textgreater}1000 ncRNAs with an evolutionarily conserved RNA secondary structure. Of these, about a quarter are located in introns of known protein coding sequences. A few RNA motifs can be identified as known RNAs, including approximately 300 tRNAs, some 100 snRNA genes and a few microRNAs and snoRNAs}

\subsubsection{Stadler2007.pdf}
\cite{Will2007}
\textit{The RFAM database defines families of ncRNAs by means of sequence similarities that are sufficient to establish homology. In some cases, such as microRNAs and box H/ACA snoRNAs, functional commonalities define classes of RNAs that are characterized by structural similarities, and typically consist of multiple RNA families. Recent advances in high-throughput transcriptomics and comparative genomics have produced very large sets of putative noncoding RNAs and regulatory RNA signals. For many of them, evidence for stabilizing selection acting on their secondary structures has been derived, and at least approximate models of their structures have been computed. The overwhelming majority of these hypothetical RNAs cannot be assigned to established families or classes. We present here a structure-based clustering approach that is capable of extracting putative RNA classes from genome-wide surveys for structured RNAs. The LocARNA (local alignment of RNA) tool implements a novel variant of the Sankoff algorithm that is sufficiently fast to deal with several thousand candidate sequences. The method is also robust against false positive predictions, i.e., a contamination of the input data with unstructured or nonconserved sequences. We have successfully tested the LocARNA-based clustering approach on the sequences of the RFAM-seed alignments. Furthermore, we have applied it to a previously published set of 3,332 predicted structured elements in the Ciona intestinalis genome (Missal K, Rose D, Stadler PF (2005) Noncoding RNAs in Ciona intestinalis. Bioinformatics 21 (Supplement 2): i77-i78). In addition to recovering, e.g., tRNAs as a structure-based class, the method identifies several RNA families, including microRNA and snoRNA candidates, and suggests several novel classes of ncRNAs for which to date no representative has been experimentally characterized.}

\subsubsection{Norden-Krichmaretal2007.pdf}
\cite{Norden-Krichmar2007}
\textit{BACKGROUND: This study reports the first collection of validated microRNA genes in the sea squirt, Ciona intestinalis. MicroRNAs are processed from hairpin precursors to {\~{}}22 nucleotide RNAs that base pair to target mRNAs and inhibit expression. As a member of the subphylum Urochordata (Tunicata) whose larval form has a notochord, the sea squirt is situated at the emergence of vertebrates, and therefore may provide information about the evolution of molecular regulators of early development.RESULTS: In this study, computational methods were used to predict 14 microRNA gene families in Ciona intestinalis. The microRNA prediction algorithm utilizes configurable microRNA sequence conservation and stem-loop specificity parameters, grouping by miRNA family, and phylogenetic conservation to the related species, Ciona savignyi. The expression for 8, out of 9 attempted, of the putative microRNAs in the adult tissue of Ciona intestinalis was validated by Northern blot analyses. Additionally, a target prediction algorithm was implemented, which identified a high confidence list of 240 potential target genes. Over half of the predicted targets can be grouped into the gene ontology categories of metabolism, transport, regulation of transcription, and cell signaling.CONCLUSION: The computational techniques implemented in this study can be applied to other organisms and serve to increase the understanding of the origins of non-coding RNAs, embryological and cellular developmental pathways, and the mechanisms for microRNA-controlled gene regulatory networks.}
\subsubsection{Fuetal2008.pdf}
\cite{Fu2008}
\textit{Recent studies reveal correlation between microRNA (miRNA) innovation and increased developmental complexity. This is exemplified by dramatic expansion of the miRNA inventory in vertebrates, a lineage where genome duplication has played a significant evolutionary role. Urochordates, the closest extant group to the vertebrates, exhibit an opposite trend to genome and morphological simplification. We show that the urochordate, larvacean, Oikopleura dioica, possesses the requisite miRNA biogenic machinery. The miRNAs isolated by small RNA cloning were expressed throughout the short life cycle, a number of which were stocked as maternal determinants prior to rapid embryonic development. We identify sex-specific miRNAs that appeared as male/female gonad differentiation became apparent and were maintained throughout spermatogenesis. Whereas 80{\%} of mammalian miRNAs are hosted in introns of protein-coding genes, the majority of O. dioica miRNA loci were located in antisense orientations to such genes. Including sister group ascidians in analysis of the urochordate miRNA repertoire, we find that 11 highly conserved bilaterian miRNA families have been lost or derived to the point they are not recognizable in urochordates and a further 4 of these families are absent in larvaceans. Subsequent to this loss/derivation, at least 29 novel miRNA families have been acquired in larvaceans. This suggests a profound reorganization of the miRNA repertoire integral to evolution in the urochordate lineage.}
\subsubsection{Hendrixetal2010.pdf}
\cite{Hendrix2010}
\textit{MicroRNAs (miRs) have been broadly implicated in animal development and disease. We developed a novel computational strategy for the systematic, whole-genome identification of miRs from high throughput sequencing information. This method, miRTRAP, incorporates the mechanisms of miR biogenesis and includes additional criteria regarding the prevalence and quality of small RNAs arising from the antisense strand and neighboring loci. This program was applied to the simple chordate Ciona intestinalis and identified nearly 400 putative miR loci.}

\subsubsection{Keshavanetal2010.pdf}

\cite{Keshavan2010}
MicroRNAs (miRNAs) are conserved non-coding small RNAs with potent post-transcriptional gene
regulatory functions. Recent computational approaches and sequencing of small RNAs had indicated
the existence of about 80 Ciona intestinalis miRNAs, although it was not clear whether other
miRNA genes were present in the genome. We undertook an alternative computational approach
to look for Ciona miRNAs. Conserved non-coding sequences from the C. intestinalis genome were
extracted and computationally folded to identify putative hairpin-like structures. After applying additional
criteria, we obtained 458 miRNA candidates whose sequences were used to design a custom
microarray. Over 100 of our predicted hairpins were identified in this array when probed with RNA
from various Ciona stages. We also compared our predictions to recently deposited sequences of
Ciona small RNAs and report that 170 of our predicted hairpins are represented in this data set.
Altogether, about 250 of our 458 predicted miRNAs were represented in either our array data or the
small-RNA sequence database. These results suggest that Ciona has a large number of genomically
encoded miRNAs that play an important role in modulating gene activity in developing
embryos and adults.


\subsubsection{Friedlander-2012.pdf}
\cite{Friedlander2012}
\textit{microRNAs (miRNAs) are a large class of small non-coding RNAs which post-transcriptionally regulate the expression of a large fraction of all animal genes and are important in a wide range of biological processes. Recent advances in high-throughput sequencing allow miRNA detection at unprecedented sensitivity, but the computational task of accurately identifying the miRNAs in the background of sequenced RNAs remains challenging. For this purpose, we have designed miRDeep2, a substantially improved algorithm which identifies canonical and non-canonical miRNAs such as those derived from transposable elements and informs on high-confidence candidates that are detected in multiple independent samples. Analyzing data from seven animal species representing the major animal clades, miRDeep2 identified miRNAs with an accuracy of 98.6-99.9{\%} and reported hundreds of novel miRNAs. To test the accuracy of miRDeep2, we knocked down the miRNA biogenesis pathway in a human cell line and sequenced small RNAs before and after. The vast majority of the {\textgreater}100 novel miRNAs expressed in this cell line were indeed specifically downregulated, validating most miRDeep2 predictions. Last, a new miRNA expression profiling routine, low time and memory usage and user-friendly interactive graphic output can make miRDeep2 useful to a wide range of researchers.}

\subsubsection{Teraietal2012.PDF}
\cite{Terai2012}
\textit{MicroRNA (miRNA) precursor hairpins have a unique secondary structure, nucleotide length, and nucleotide content that are in most cases evolutionarily conserved. The aim of this study was to utilize position-specific features of miRNA hairpins to improve their identification. To this end, we defined the evolutionary and structurally conserved features in each position of miRNA hairpins with heuristically derived values, which were successfully integrated using a probabilistic framework. Our method, miRRim2, can not only accurately detect miRNA hairpins, but infer the location of a mature miRNA sequence. To evaluate the accuracy of miRRim2, we designed a cross validation test in which the whole human genome was used for evaluation. miRRim2 could more accurately detect miRNA hairpins than the other computational predictions that had been performed on the human genome, and detect the position of the 5'-end of mature miRNAs with sensitivity and positive predictive value (PPV) above 0.4. To further evaluate miRRim2 on independent data, we applied it to the Ciona intestinalis genome. Our method detected 47 known miRNA hairpins among top 115 candidates, and pinpointed the 5'-end of mature miRNAs with sensitivity and PPV about 0.4. When our results were compared with deep-sequencing reads of small RNA libraries from Ciona intestinalis cells, we found several candidates in which the predicted mature miRNAs were in good accordance with deep-sequencing results.}


\subsubsection{Kusakabeetal2013.pdf}

\cite{Kusakabe2013}
\textit{Muscle-specific miR-1/206 and miR-133 families have been suggested to play fundamental roles in skeletal and cardiac myogenesis in vertebrates. To gain insights into the relationships between the divergence of these miRs and muscular tissue types, we investigated the expression patterns of miR-1 and miR-133 in two ascidian Ciona species and compared their genomic structures with those of other chordates. We found that Ciona intestinalis and Ciona savignyi each possess a single copy of the miR-1/miR-133 cluster, which is only 350 nucleotide long. During embryogenesis, Ciona miR-1 and miR-133 are generated as a single continuous primary transcript accumulated in the nuclei of the tail muscle cells, starting at the gastrula stage. In adults, mature miR-133 and miR-1 are differentially expressed in the heart and body wall muscle. Expression of the reporter gene linked to the 850-bp upstream region of the predicted transcription start site confirmed that this region drives the muscle-specific expression of the primary transcript of miR-1/miR-133. In many deuterostome lineages, including that of Ciona, the miR-1/133 cluster is located in the same intron of the mind bomb (mib) gene in reverse orientation. Our results suggest that the origin of genomic organization and muscle-specific regulation of miR-1/133 can be traced back to the ancestor of chordates. Duplication of this miR cluster might have led to the remarkable elaboration in the morphology and function of skeletal muscles in the vertebrate lineage. {\textcopyright} 2012 Elsevier B.V. All rights reserved.}

\subsubsection{Jueetal2016.pdf}
\cite{Jue2016}
\textit{A preliminary genome sequence has been assembled for the Southern Ocean salp, Salpa thompsoni (Urochordata, Thaliacea). Despite the ecological importance of this species in Antarctic pelagic food webs and its potential role as an indicator of changing Southern Ocean ecosystems in response to climate change, no genomic resources are available for S. thompsoni or any closely-related urochordate species. Using a multiple-platform, multiple-individual approach, we have produced a 318,767,936 bp genome sequence, covering more than 50{\%} of the estimated 602 MB (±173 MB) genome size for S. thompsoni Using a non-redundant set of predicted proteins, more than 50{\%} (16,823) of sequences showed significant homology to known proteins and {\~{}}38{\%} (12,151) of the total protein predictions were associated with Gene Ontology functional information. We have generated 109,958 SNP variant and 9,782 indel predictions for this species, serving as a resource for future phylogenomic and population genetic studies. Comparing the salp genome to available assemblies for four other urochordates, Botryllus schlosseri, Ciona intestinalis, Ciona savignyi and Oikopleura dioica, we found that S. thompsoni shares the previously-estimated rapid rates of evolution for these species. High mutation rates are thus independent of genome size, suggesting that rates of evolution {\textgreater}1.5 times that observed for vertebrates are a broad taxonomic characteristic of urochordates. Tests for positive selection implemented in PAML revealed a small number of genes with sites undergoing rapid evolution, including genes involved in ribosome biogenesis and metabolic and immune process that may be reflective of both adaptation to polar, planktonic environments as well as the complex life history of the salps. Finally, we performed an initial survey of small RNAs, revealing the presence of known, conserved miRNAs, as well as novel miRNA genes; unique piRNAs; and mature miRNA signatures for varying developmental stages. Collectively, these resources provide a genomic foundation supporting S. thompsoni as a model species for further examination of the exceptional rates and patterns of genomic evolution shown by urochordates. Additionally, genomic data will allow for the development of molecular indicators of key life history events and processes and afford new understandings and predictions of impacts of climate change on this key species of Antarctic pelagic ecosystems.}


\subsubsection{Wangetal2017.pdf}
\cite{Wang2017}
\textit{Background: miRNAs play essential roles in the modulation of cellular functions via degradation and/or translation
attenuation of target mRNAs. They have been surveyed in a single ascidian genus, Ciona. Recently, an annotated
draft genome sequence for a distantly related ascidian, Halocynthia roretzi, has become available, but miRNAs in
H. roretzi have not been previously studied.
Results: We report the prediction of 319 candidate H. roretzi miRNAs, obtained through three complementary
methods. Experimental validation suggests that more than half of these candidate miRNAs are expressed during
embryogenesis. The majority of predicted H. roretzi miRNAs appear specific to ascidians or tunicates, and only 32
candidates, belonging to 25 families, are widely conserved across metazoans.
Conclusion: Our study presents a comprehensive identification of candidate H. roretzi miRNAs. This resource
will facilitate the study of the mechanisms for miRNA-controlled gene regulatory networks during ascidian
development. Further, our analysis suggests that the majority of Halocynthia miRNAs are specific to ascidian
or tunicates, with only a small number of widely conserved miRNAs. This result is consistent with the general
notion that animal miRNAs are less conserved between taxa than plant ones.}


\subsection{miRNA in an perspective evolution}

\subsubsection{Introduction}

\subsubsection{Pignatelli2016.pdf}
\cite{Pignatelli2016}
\textit{Annotation of orthologous and paralogous genes is necessary for many aspects of evolutionary analysis. Methods to infer these homology relationships have traditionally focused on protein-coding genes and evolutionary models used by these methods normally assume the positions in the protein evolve independently. However, as our appreciation for the roles of non-coding RNA genes has increased, consistently annotated sets of orthologous and paralogous ncRNA genes are increasingly needed. At the same time, methods such as PHASE or RAxML have implemented substitution models that consider pairs of sites to enable proper modelling of the loops and other features of RNA secondary structure. Here, we present a comprehensive analysis pipeline for the automatic detection of orthologues and paralogues for ncRNA genes. We focus on gene families represented in Rfam and for which a specific covariance model is provided. For each family ncRNA genes found in all Ensembl species are aligned using Infernal, and several trees are built using different substitution models. In parallel, a genomic alignment that includes the ncRNA genes and their flanking sequence regions is built with PRANK. This alignment is used to create two additional phylogenetic trees using the neighbour-joining (NJ) and maximum-likelihood (ML) methods. The trees arising from both the ncRNA and genomic alignments are merged using TreeBeST, which reconciles them with the species tree in order to identify speciation and duplication events. The final tree is used to infer the orthologues and paralogues following Fitch's definition. We also determine gene gain and loss events for each family using CAFE. All data are accessible through the Ensembl Comparative Genomics ('Compara') API, on our FTP site and are fully integrated in the Ensembl genome browser, where they can be accessed in a user-friendly manner.Database URL: http://www.ensembl.org.}
\subsubsection{Dai2009.pdf}

\cite{Dai2009}
\textit{Cephalochordates, urochordates, and vertebrates comprise the three extant groups of chordates. Although higher morphological and developmental similarity exists between cephalochordates and vertebrates, molecular phylogeny studies have instead suggested that the morphologically simplified urochordates are the closest relatives to vertebrates. MicroRNAs (miRNAs) are regarded as the major factors driving the increase of morphological complexity in early vertebrate evolution, and are extensively characterized in vertebrates and in a few species of urochordates. However, the comprehensive set of miRNAs in the basal chordates, namely the cephalochordates, remains undetermined. Through extensive sequencing of a small RNA library and genomic homology searches, we characterized 100 miRNAs from the cephalochordate amphioxus, Branchiostoma japonicum, and B. floridae. Analysis of the evolutionary history of the cephalochordate miRNAs showed that cephalochordates possess 54 miRNA families homologous to those of vertebrates, which is threefold higher than those shared between urochordates and vertebrates. The miRNA contents demonstrated a clear correlation between the extent of miRNA overlapping and morphological similarity among the three chordate groups, providing a strong evidence of miRNAs being the major genetic factors driving morphological complexity in early chordate evolution.}
\subsubsection{Candini2012.pdf}
\cite{Candiani2012}
\textit{MicroRNAs (miRNAs) are small non-coding RNAs that negatively regulate gene expression and thus control diverse biological processes. The high interest in miRNAs as an important mediator of post-transcriptional gene regulation has led to the discovery of miRNAs in several organisms. The present article outlines and discusses the current status of miRNAs information on the basal chordate amphioxus and the evolution of miRNAs in metazoans.}

\subsubsection{Stadler2006.pdf}
\cite{Hertel2006}
\textit{UNLABELLED: Recently, genome-wide surveys for non-coding RNAs have provided evidence for tens of thousands of previously undescribed evolutionary conserved RNAs with distinctive secondary structures. The annotation of these putative ncRNAs, however, remains a difficult problem. Here we describe an SVM-based approach that, in conjunction with a non-stringent filter for consensus secondary structures, is capable of efficiently recognizing microRNA precursors in multiple sequence alignments. The software was applied to recent genome-wide RNAz surveys of mammals, urochordates, and nematodes. AVAILABILITY: The program RNAmicro is available as source code and can be downloaded from http://www.bioinf.uni-leipzig/Software/RNAmicro.}

\subsubsection{SamGriffit2007.pdf}
\cite{Griffiths-Jones2011}
\textit{MicroRNAs (miRNAs) modulate transcript stability and translation. Functional mature miRNAs are processed from one or both arms of the hairpin precursor. The miR-100/10 family has undergone three independent evolutionary events that have switched the arm from which the functional miRNA is processed. The dominant miR-10 sequences in the insects Drosophila melanogaster and Tribolium castaneum are processed from opposite arms. However, the duplex produced by Dicer cleavage has an identical sequence in fly and beetle. Expression of the Tribolium miR-10 sequence in Drosophila S2 cells recapitulates the native beetle pattern. Thus, arm usage is encoded in the primary miRNA sequence, but outside the mature miRNA duplex. We show that the predicted messenger RNA targets and inferred function of sequences from opposite arms differ significantly. Arm switching is likely to be general, and provides a fundamental mechanism to evolve the function of a miRNA locus and target gene network.},
author = {Griffiths-Jones, Sam and Hui, Jerome H L and Marco, Antonio and Ronshaugen, Matthew}

\subsubsection{Hertel2015.pdf}
\cite{Hertel2015}
\textit{MicroRNAs are important regulatory small RNAs in many eukaryotes. Due to their small size and simple structure, they are readily innovated de novo. Throughout the evolution of animals, the emergence of novel microRNA families traces key morphological innovations. Here, we use a computational approach based on homology search and parsimony-based presence/absence analysis to draw a comprehensive picture of microRNA evolution in 159 animal species. We confirm previous observations regarding bursts of innovations accompanying the three rounds of genome duplications in vertebrate evolution and in the early evolution of placental mammals. With a much better resolution for the invertebrate lineage compared to large-scale studies, we observe additional bursts of innovation, e.g., in Rhabditoidea. More importantly, we see clear evidence that loss of microRNA families is not an uncommon phenomenon. The Enoplea may serve as a second dramatic example beyond the tunicates. The large-scale analysis presented here also highlights several generic technical issues in the analysis of very large gene families that will require further research.}

\subsubsection{Yanetal2014.pdf}

\cite{Yang2014}
\textit{Mir-181 is an ancient microRNA (miRNA) gene family that originated in urochordata. Although their functions were subjected to extensive studies in recent years, their evolutionary process remains largely unknown. Here we systematically investigated the homologous genes of the mir-181 family by a sequence similarity search. Representative sequences of the mir-181 gene family were used to reconstruct their evolutionary history. Our results indicated that this family could have derived from multiple duplications, which include two rounds of whole genome duplications and one round of segmental replication. Functional annotation of the target genes of the mir-181 family suggested that this family could participate in some important biological processes including transcriptional and translational regulation, signaling transduction etc. This analysis presented a complex evolutionary dynamics for the origination of a miRNA gene family. {\textcopyright} 2014 Elsevier Ltd.}


\section{Other ncRNAs associated to development}
\subsection{YC RNA}
\subsubsection{Swalla1995.pdf}

\cite{Swalla1995}
\textit{A cDNA library prepared from one-cell zygotes of the ascidian Styela 
clava was screened with probes from isolated cellular fractions to identify 
clones encoding RNAs localized in the yellow crescent or myoplasm, a 
cytoskeletal domain with multiple developmental roles. The differential screen 
yielded five overlapping cDNA (Styela clava yellow crescent or ScYC) clones 
encoding a 1.2-kb polyadenylated RNA (yellow crescent or YC RNA) which is 
present throughout embryonic development. In situ hybridization confirmed that 
YC RNA is localized in the yellow crescent. Antisense probes containing the 3' 
region of YC RNA hybridize with multiple maternal and zygotic RNAs, suggesting 
sequence homologies with other transcripts. YC RNA was first detected during 
oogenesis when transcripts accumulate in the perinuclear region of vitellogenic 
oocytes and are gradually translocated to the cortex. The YC transcripts are 
localized in the cortex of unfertilized eggs but after fertilization segregate 
with the myoplasm to the yellow crescent. During cleavage most YC transcripts 
enter the primary muscle cell lineage. YC RNA is also present in the secondary 
muscle cells. The YC transcripts are retained in the myoplasm of oocytes and 
eggs extracted with the non-ionic detergent Triton X-100, suggesting that they 
are associated with the cytoskeleton. The nucleotide sequence of the longest 
ScYC clone contains a short open reading frame (ORF). The YC ORF would encode a 
putative polypeptide of 49 amino acids, which shows no significant homology to 
known proteins. Several features of the YC RNA, however, suggest that it 
functions as an RNA rather than as a protein coding molecule. We conclude that 
the myoplasm contains a novel maternal RNA which is associated with the 
cytoskeleton and segregated to the muscle cells during ascidian embryogenesis. 
The YC RNA may be a new member of a growing family of noncoding RNAs that play 
important roles in growth and development.}

\subsection{moRNAs}
\subsubsection{Bortolluzzietal2011.pdf}
\cite{Bortoluzzi2011a}
\textit{Recent studies have exponentially increased the number of
known noncoding RNA categories, including microRNA
(miRNA), small interfering RNA (siRNA), PIWI elementinteracting
RNA and various classes of long noncoding
RNA (ncRNA), that fulfill key roles as transcriptional
and post-transcriptional regulators and guides of chromatin-
modifying complexes [1]. Among these short RNAs,
miRNAs are post-transcriptional regulators of gene expression
in a wide range of biological processes and diseases
[2]. miRNAs are considered as prominent tumour
markers, relevant targets for therapy and therapeutic
agents [3]. Here, we discuss the discovery of a novel type
of miRNA-related small RNA, miRNA–offset RNA
(moRNA), whose function is currently unknown.}

\subsubsection{Shietal2009.pdf}

\cite{Shi2009}
\textit{MicroRNAs (miRNAs) have been implicated in various cellular processes. They are thought to function primarily as inhibitors of gene activity by attenuating translation or promoting mRNA degradation. A typical miRNA gene produces a predominant approx21-nucleotide (nt) RNA (the miRNA) along with a less abundant miRNA* product. We sought to identify miRNAs from the simple chordate Ciona intestinalis through comprehensive sequencing of small RNA libraries created from different developmental stages. Unexpectedly, half of the identified miRNA loci encode up to four distinct, stable small RNAs. The additional RNAs, miRNA-offset RNAs (moRs), are generated from sequences immediately adjacent to the predicted approx60-nt pre-miRNA. moRs seem to be produced by RNAse III–like processing, are approx20 nt long and, like miRNAs, are observed at specific developmental stages. We present evidence suggesting that the biogenesis of moRs results from an intrinsic property of the miRNA processing machinery in C. intestinalis.}

\subsection{RMST9}
At the end of out paper “Long non-coding RNAs
The scope of lncRNA annotations by homology is very limited due to their low levels of sequence conservation. The Rfam database therefore lists only a small number of well-conserved elements. The HMM-based search identified a plausible homolog of RMST 9, the conserved region 9 of the Rhabdomyosarcoma 2 associated transcript, which has been associated with neurogenesis processes by its interaction with SOX2 [38]. To check whether this surprising hit is likely to be a true positive we also investigated the genomes of C. intestinalis, C. savignyi, B. schlosseri and B. floridae and found putative homologs with p<10−3 and cmsearch identifies these sequences with E<10−9. The corresponding multiple sequence alignment can be found in Additional file 5: S8. At least parts of the RMST lncRNA are thus conserved across chordates, making it one of the best conserved lncRNAs.

\subsubsection{ShiYan2013.pdf}
\cite{Bogu2013}
\textit{Long noncoding RNAs (lncRNAs) are abundant in the mammalian transcriptome, and many are specifically expressed in the brain. We have identified a group of lncRNAs, including rhabdomyosarcoma 2-associated transcript (RMST), which are indispensable for neurogenesis. Here, we provide mechanistic insight into the role of human RMST in modulating neurogenesis. RMST expression is specific to the brain, regulated by the transcriptional repressor REST, and increases during neuronal differentiation, indicating a role in neurogenesis. RMST physically interacts with SOX2, a transcription factor known to regulate neural fate. RMST and SOX2 coregulate a large pool of downstream genes implicated in neurogenesis. Through RNA interference and genome-wide SOX2 binding studies, we found that RMST is required for the binding of SOX2 to promoter regions of neurogenic transcription factors. These results establish the role of RMST as a transcriptional coregulator of SOX2 and a key player in the regulation of neural stem cell fate. ?? 2013 Elsevier Inc.}

\subsection{SL RNA}
\subsubsection{Philippe2004SLRNA.pdf}
\cite{Ganot2004}
\textit{trans splicing of a spliced-leader RNA (SL RNA) to the 5-ends of mRNAs has been shown to have a limited
and sporadic distribution among eukaryotes. Within metazoans, only nematodes are known to process polycistronic
pre-mRNAs, produced from operon units of transcription, into mature monocistronic mRNAs via an
SL RNA trans-splicing mechanism. Here we demonstrate that a chordate with a highly compact genome,
Oikopleura dioica, now joins Caenorhabditis elegans in coupling trans splicing with processing of polycistronic
transcipts. We identified a single SL RNA which associates with Sm proteins and has a trimethyl guanosine
cap structure reminiscent of spliceosomal snRNPs. The same SL RNA, estimated to be trans-spliced to at least
25\% of O. dioica mRNAs, is used for the processing of both isolated or first cistrons and downstream cistrons
in a polycistronic precursor. Remarkably, intercistronic regions in O. dioica are far more reduced than those
in either nematodes or kinetoplastids, implying minimal cis-regulatory elements for coupling of 3-end formation
and trans splicing.}


\subsubsection{LeBlanc1989.pdf}

\cite{LEBLANC1989}
\textit{We have identified the sea urchin cognate of the mammalian signal recognition particle (SRP).
This particle contains the diagnostic 7 SL small RNA, sediments at a similar velocity to that
reported for the mammalian particle, and is found associated with the ER and polysomes. We
have examined its subcellular localization during embryogenesis in order to determine whether
it could serve in a translational regulatory capacity for a subset of the stored maternal mRNAs.
In these studies the 7 SL RNA was used as a marker for the particle, since we determined that
the 7 SL RNA exists exclusively within the SRPlike
particle at all developmental stages. The
relative distribution of the SRP among cytoplasmic structures changes dramatically during
development. This represents an actual change in subcellular localization because the 7 SL
RNA level remains nearly constant per embryo until the pluteus stage, when it increases
slightly. In eggs, the SRP exists almost entirely free in the cytoplasm as an 11 S particle. Very
soon after fertilization and throughout development there is an increase in the association of the particle with rapidly sedimenting structures, until by the pluteus stage greater than 90\% of the
SRP exists in a bound state. The nature of the associations is complex, and the bound
structures include, at least in part, ribosomes, polysomes, and microsomes. The SRP is
associated with microsomal membranes in gastrula (36 hr) but not in blastula (12 hr) or earlier
embryos. Using the criteria of sensitivity to Triton X100,
we determined that 16\% of the SRP in
a 10,000g cytoplasmic fraction was bound to membranes in a microsomal (endoplasmic
reticulum)containing
fraction in the gastrula. In contrast, less than 1\% was membrane
associated in the blastula. The SRP was also found in a ribosomepolysome
fraction in 12, 36, and 48hr embryos, but not in eggs. Finally, a small but significant portion of the SRP was found
associated with monosomes in cleavage stage embryos. The possible role the SRP could play
in the elongation arrest of stored maternal messages for secreted proteins is discussed.}

